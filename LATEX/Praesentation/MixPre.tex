\documentclass[10pt]{beamer}

% proper umlauts
\usepackage[utf8]{inputenc}
\usepackage[T1]{fontenc}

% german
\usepackage[ngerman]{babel}

% AMS math for formulae
\usepackage{amsmath}
\usepackage{amssymb}
\usepackage{amsfonts}

% graphics
\usepackage{graphicx}

% some other useful packages
%\usepackage{url}


%%%%%%%%%%%%%%%%%%%%%%%%%%%%%%%%%%%%%%%%%%%%%%%%%%%%%%%%%%%%%%%%%%%%%
%%% my shortcuts
%%%%%%%%%%%%%%%%%%%%%%%%%%%%%%%%%%%%%%%%%%%%%%%%%%%%%%%%%%%%%%%%%%%%%
\newcommand{\highlight}[1]{{\color{blue}\bf#1}}
\newcommand{\todo}[1]{{\color{blue}\bf TODO: #1}}


%%%%%%%%%%%%%%%%%%%%%%%%%%%%%%%%%%%%%%%%%%%%%%%%%%%%%%%%%%%%%%%%%%%%%
%%% setup beamer presentation theme
%%%%%%%%%%%%%%%%%%%%%%%%%%%%%%%%%%%%%%%%%%%%%%%%%%%%%%%%%%%%%%%%%%%%%

\mode<presentation>
{
  \usetheme{TUDortmund2}
}

% include intermediate TOCs automatically at each \section
\AtBeginSection[]
{
  \begin{frame}[c]
    \frametitle{Content}
    \tableofcontents[currentsection]
  \end{frame}
}

% Suppress navigation symbols
\setbeamertemplate{navigation symbols}{}


%%%%%%%%%%%%%%%%%%%%%%%%%%%%%%%%%%%%%%%%%%%%%%%%%%%%%%%%%%%%%%%%%%%%%
%%% titlepage information
%%%%%%%%%%%%%%%%%%%%%%%%%%%%%%%%%%%%%%%%%%%%%%%%%%%%%%%%%%%%%%%%%%%%%

\title{Mixed Precision}
\author{Christoph Höppke, Daniel Thomaschewsik}
\institute[TU Dortmund]{TU Dortmund}
\date{Version:\today}



%%%%%%%%%%%%%%%%%%%%%%%%%%%%%%%%%%%%%%%%%%%%%%%%%%%%%%%%%%%%%%%%%%%%%
%%% titlepage
%%%%%%%%%%%%%%%%%%%%%%%%%%%%%%%%%%%%%%%%%%%%%%%%%%%%%%%%%%%%%%%%%%%%%
\begin{document}
% Use non-transparent version of logo for title page
\logo{\centering%
\includegraphics[height=0.5cm]{figures/logo_TUDortmund}%
\hspace*{1em}%
\includegraphics[height=0.5cm]{figures/logo_fakm}%
\hspace*{15em}}
\begin{frame}[c]
  \titlepage
\end{frame}

% no logo from now on, just eats space
\logo{}

\begin{frame}{Inhaltsübersicht}
\tableofcontents

\end{frame}
%%%%%%%%%%%%%%%%%%%%%%%%%%%%%%%%%%%%%%%%%%%%%%%%%%%%%%%%%%%%%%%%%%%%%
%%%%%%%%%%%%%%%%%%%%%%%%%%%%%%%%%%%%%%%%%%%%%%%%%%%%%%%%%%%%%%%%%%%%%
\section{What is a Mixed Precision Method?}
%%%%%%%%%%%%%%%%%%%%%%%%%%%%%%%%%%%%%%%%%%%%%%%%%%%%%%%%%%%%%%%%%%%%%
%%%%%%%%%%%%%%%%%%%%%%%%%%%%%%%%%%%%%%%%%%%%%%%%%%%%%%%%%%%%%%%%%%%%%
\subsection{Definition}
\begin{frame}[c]

\end{frame}


\subsection{Performance Gains}
\begin{frame}

\end{frame}

\subsection{Precision}
\begin{frame}{Roundoff and Cancellation}

\end{frame}

\begin{frame}{Computational Precision vs Accuracy of Result}

\end{frame}

\subsection{Data Error and Truncation}
\begin{frame}{DataError and Truncation}
\end{frame}


\subsection{Floting Point Operations. A deeper analysis}
\begin{frame}{}

\end{frame}

\section{History of Mixed Preceition in the context of GPGPU calculations}
\begin{frame}{History of Mixed Preceition in the context of GPGPU calculations}

\end{frame}

\section{Why Mixed Precision is difficult}
\begin{frame}{Why Mixed Precision is difficult}

\end{frame}

\section{Unconventional computation Hardware}
\begin{frame}{Unconventional compuation Hardware}

\end{frame}

\section{Iterative Refinement}

\subsection{Testresults}

\end{document}



\grid
