\documentclass[10pt]{beamer}

% proper umlauts
\usepackage[utf8]{inputenc}
\usepackage[T1]{fontenc}

% german
\usepackage[ngerman]{babel}

% AMS math for formulae
\usepackage{amsmath}
\usepackage{amssymb}
\usepackage{amsfonts}

% graphics
\usepackage{graphicx}

% some other useful packages
%\usepackage{url}


%%%%%%%%%%%%%%%%%%%%%%%%%%%%%%%%%%%%%%%%%%%%%%%%%%%%%%%%%%%%%%%%%%%%%
%%% my shortcuts
%%%%%%%%%%%%%%%%%%%%%%%%%%%%%%%%%%%%%%%%%%%%%%%%%%%%%%%%%%%%%%%%%%%%%
\newcommand{\highlight}[1]{{\color{blue}\bf#1}}
\newcommand{\todo}[1]{{\color{blue}\bf TODO: #1}}


%%%%%%%%%%%%%%%%%%%%%%%%%%%%%%%%%%%%%%%%%%%%%%%%%%%%%%%%%%%%%%%%%%%%%
%%% setup beamer presentation theme
%%%%%%%%%%%%%%%%%%%%%%%%%%%%%%%%%%%%%%%%%%%%%%%%%%%%%%%%%%%%%%%%%%%%%

\mode<presentation>
{
  \usetheme{TUDortmund2}
}

% include intermediate TOCs automatically at each \section
\AtBeginSection[]
{
  \begin{frame}[c]
    \frametitle{Content}
    \tableofcontents[currentsection]
  \end{frame}
}

% Suppress navigation symbols
\setbeamertemplate{navigation symbols}{}


%%%%%%%%%%%%%%%%%%%%%%%%%%%%%%%%%%%%%%%%%%%%%%%%%%%%%%%%%%%%%%%%%%%%%
%%% titlepage information
%%%%%%%%%%%%%%%%%%%%%%%%%%%%%%%%%%%%%%%%%%%%%%%%%%%%%%%%%%%%%%%%%%%%%

\title{Mixed Precision}
\author{Christoph Höppke, Daniel Thomaschewsik}
\institute[TU Dortmund]{TU Dortmund}
\date{Version:\today}



%%%%%%%%%%%%%%%%%%%%%%%%%%%%%%%%%%%%%%%%%%%%%%%%%%%%%%%%%%%%%%%%%%%%%
%%% titlepage
%%%%%%%%%%%%%%%%%%%%%%%%%%%%%%%%%%%%%%%%%%%%%%%%%%%%%%%%%%%%%%%%%%%%%
\begin{document}
% Use non-transparent version of logo for title page
\logo{\centering%
\includegraphics[height=0.5cm]{figures/logo_TUDortmund}%
\hspace*{1em}%
\includegraphics[height=0.5cm]{figures/logo_fakm}%
\hspace*{15em}}
\begin{frame}[c]
  \titlepage
\end{frame}

% no logo from now on, just eats space
\logo{}

\begin{frame}{Content}
\tableofcontents

\end{frame}
%%%%%%%%%%%%%%%%%%%%%%%%%%%%%%%%%%%%%%%%%%%%%%%%%%%%%%%%%%%%%%%%%%%%%
%%%%%%%%%%%%%%%%%%%%%%%%%%%%%%%%%%%%%%%%%%%%%%%%%%%%%%%%%%%%%%%%%%%%%
\section{What is a Mixed Precision Method?}
%%%%%%%%%%%%%%%%%%%%%%%%%%%%%%%%%%%%%%%%%%%%%%%%%%%%%%%%%%%%%%%%%%%%%
%%%%%%%%%%%%%%%%%%%%%%%%%%%%%%%%%%%%%%%%%%%%%%%%%%%%%%%%%%%%%%%%%%%%%
\subsection{Definition}
\begin{frame}{Definition}

\begin{block}{\textbf{Definition:}}
\textbf{An Algorithm that 
uses different precisions in its computation}
\end{block}

\textbf{Example:} \color{red}{double(a)} \color{black}+ \color{blue}{float(b)}\color{black}
~\\~\\
\textbf{Goal: } 
\begin{itemize}
\item[] Obtain the \color{red}{same accuracy}\color{black}~by using 
high precision
\item[] but \color{blue}{better performance} \color{black}
by utilizing low precision computations
\end{itemize}
\end{frame}


\subsection{Performance Gains}
\begin{frame}{Performance Gains}
\begin{itemize}
\item \textbf{Bandwidth bound algorithm}
	\begin{itemize}
	\item 64 bit = 1 double = 2 floats = 4 halfs
	\item More variables per \color{red}{bandwidth}\color{black}
	\item More variables per \color{red}{storage}\color{black}
	\item Applies to all memory levels: network, disc, main, device, 
			local, register
	\end{itemize}
\item \textbf{Computation bound algorithm}
	\begin{itemize}
	\item 1 double addition $\approx$ 2 float additions $\approx$ 4 half 
			additions (linear)
	\item 1 double multip. $\approx$ 4 float multip. $\approx$ 16 
			half multip. (quadratic)
	\item $\rightarrow$ up to 16 times better computational efficiency
	\end{itemize}
\end{itemize}
\end{frame}

\subsection{Precision}
\begin{frame}{Roundoff and Cancellation}

\begin{block}{Definition Machine Precision}
The smalles positive number $\epsilon$ for wich a floating point 
calculations evaluates the expression $1 + \epsilon > 1$ to be true.
\end{block}

\textbf{Examples:}
\begin{itemize}
\item $\epsilon_{double} \approx 2.220446049250313\cdot 10^{-16}$
\item $\epsilon_{float} \approx 1.1920929\cdot 10^{-7}$
\item $\epsilon_{half} \approx 9.765625\cdot 10^{-4}$
\item So \color{red}more precision\color{black}~is usually \color{red}better
\end{itemize}
\textbf{Cancellation}
\begin{align*}
&\text{additive roundoff } & a=1+0.00000004 &= 1.00000004 &=_{fl} 1\\
&\text{multiplicative roundoff } & b = 1.0002 \cdot 0.9998 &=0.99999996 &=_{fl} 1\\
&\text{\color{blue}cancellation\color{black}~}  &c\in \lbrace a,b\rbrace 
\hspace{1cm}\pm 4 &= (c-1)\cdot 10^8 &=_{fl} 0 \\
&\text{oder~of~operations}  & 1+0.00000004 -1 &=_{fl} 0\\
& &1-1 + 0.00000004 &=_{fl} 0.0000004
\end{align*}
\end{frame}


\begin{frame}{Computational Precision vs Accuracy of Result}
\begin{block}{Instructive Example [S.M. Rump, 1988]}
$f(x,y) = (333.75 - x^2)y^6 + x^2(11x^2y^2 - 121y^4-2) + 5.5y^8 + 0.5x/y$\\
$x_0 = 77617, y_0 = 33096$
\end{block}
\begin{align*}
&\text{float s23e8} \hfill &1.1726\\
&\text{double s52e11} \hfill &1.17260394005318\\
&\text{quad s63e15} \hfill &1.172603940053178631
\end{align*}
\textbf{The correct result is:}\\
\begin{center}
\color{red}-0.82739605994682136814116509547981629\color{black}
\end{center}
\end{frame}

\subsection{Data Error and Truncation}
\begin{frame}{DataError and Truncation}
\begin{itemize}
\item Data error occurs when the \color{red} exact value~\color{black} has to be \color{red} truncated \color{black} for storage in the binary format
	\begin{itemize}
		\item $\pi$ , $\sqrt{2}$, $sin(2)$, $e^2$, $1/3$
		\item every rational number with a denominator that has a prime 
		      factor other than $2$
	\end{itemize}
\item How can float be better than double?
\begin{itemize}
	\item There is \color{red} no data error \color{black} in the operands
	\item The errors can \color{red} cancel out \color{black} themselves 
		  favorably
\end{itemize}
\color{black}
\end{itemize}

\end{frame}


\subsection{Floting Point Operations. A deeper analysis}
\begin{frame}{Floating Point Operations. A deeper analysis}
\begin{itemize}
\item Number representation
\begin{itemize}
\item half s10e5~a=$\vert$1 bit sign $s_a~|~$10 bit mantissa $m_a~|~$5 bit exp. $e_a$ 
\item float s23e8~b = $\vert$1 bit sign $s_b~|~$23 bit mantissa $m_b~|~$8 bit exp. $e_b$ 
\item double s52e11~c = $\vert$1 bit sign $s_c~|~$52 bit mantissa $m_c~|~$11 bit exp. $e_c$ 
\end{itemize}
\item Multiplication $float(a) \cdot float(b)$
	\begin{itemize}
	\item Operations: $s_a \cdot s_b$,$m_a \cdot m_b$,$e_a \cdot e_b$
	\item Excat format: s46e9 = s23e8 $\cdot$s23e8
	\item Main error: Mantissa truncated from 46 bit to 23 bit
	\end{itemize}
\item Addition $float(a) + float(b)$
	\begin{itemize}
	\item Operations: $e_{diff} = e_a - e_b$, $m_a + (m_b >> e_{diff})$, 
			normalize
	\item Exact format: s278e8 = s23e8+s23e8
	\item Main error: Maintissa truncated from 278 bit to 23 bit
	\end{itemize}
\end{itemize}
\end{frame}

\section{History of Mixed Precision in the context of GPGPU calculations}
\subsection{History of GPGPU}
\begin{frame}{History of GPGPU}

	\textbf{1999: NVIDIA GeForce 256}\\
	\begin{itemize}
	\item Term GPU was coined during the launch
	\item GPGPU calculations were only achievable by "Hacking the GPU" 
			(verry cumbersome and error-prone)
	\end{itemize}
~\\~\\
\textbf{2003-2007 First wave of GPU computing}
\begin{itemize}
\item Floating point support
\item Performance improvements
\item Geforce 8800GT Release (29.October 2007)\\
	first CUDA enabled Consumer GPU
\item Double precision was not available on the GPU
\item Mixed Precision was the ONLY way to utilize GPU horesepower
	without precision compromises
\item Mixed Precision lead to a speedup of factor 3-5 in FEM applications %[http://www.mathematik.tu-dortmund.de/~goeddeke/pubs/pdf/Strzodka_2006_MPM.pdf]
\end{itemize}
\end{frame}

\subsection{Why Mixed Precision is difficult}
\begin{frame}{Why Mixed Precision is difficult}
\begin{itemize}
\item Data has to be transferred to the GPU over the relativly \color{red} slow PCIe \color{black} interface
\item A lot of time is wasted while waiting for data
\end{itemize}
\end{frame}

\subsection{Unconventional computation Hardware}
\begin{frame}{Unconventional compuation Hardware}
\textbf{NVIDIA Jetson TK1 Bord}
\begin{itemize}
\item CPU and GPU share the same memory and use the same memory interface
\item Copying data from CPU to GPU and vice versa is much faster
\end{itemize}
\end{frame}

\section{Example Calculation}
\begin{frame}{Example Calculation}
\begin{block}{Problem}
Solve the Poission problem $- \Delta u = 1,~x\in\Omega$ with dirichlet boundry conditions $u \equiv 0,~x\in \partial \Omega$ using conforming quadrilateral 
elements for the finite-element discretization of the unit square $\Omega = 
(-1,1)^2$
\end{block}

\begin{itemize}
\item Method 1: Using a CG solver
\item Method 2 : Using a more sophisticated MG solver
\end{itemize}

\end{frame}

\subsection{Testresults}
\begin{frame}{Testresults pending}
\begin{center}
\color{red} Imagine a nice Plot\color{black}
\end{center}
\end{frame}
\end{document}